% !TEX encoding = UTF-8 Unicode
% XeLaTeX can use any Mac OS X font. See the setromanfont command below.
% Input to XeLaTeX is full Unicode, so Unicode characters can be typed directly into the source.

% The next lines tell TeXShop to typeset with xelatex, and to open and save the source with Unicode encoding.

%!TEX TS-program = xelatex

\documentclass[12pt]{article}
\usepackage[bottom=3cm]{geometry}                % See geometry.pdf to learn the layout options. There are lots.
\geometry{letterpaper}                   % ... or a4paper or a5paper or ... 
%\geometry{landscape}                % Activate for for rotated page geometry
%\usepackage[parfill]{parskip}    % Activate to begin paragraphs with an empty line rather than an indent
\usepackage{graphicx}
\usepackage{amssymb}
\usepackage{calc}
\usepackage{graphicx}
\pagestyle{empty}


%CJK Font Setup
\usepackage[PunctStyle=kaiming,AutoFakeBold=false,AutoFakeSlant=false]{xeCJK}
\setCJKmainfont[BoldFont={Songti SC Bold},ItalicFont={KaiTi}] {SimSun}
\setCJKsansfont[BoldFont={Heiti SC Medium}]{Heiti SC}
\setCJKmonofont[BoldFont={Hiragino Sans GB W6}]{Hiragino Sans GB W3}

% Will Robertson's fontspec.sty can be used to simplify font choices.
% To experiment, open /Applications/Font Book to examine the fonts provided on Mac OS X,
% and change "Hoefler Text" to any of these choices.

%English Font Setup
\usepackage{fontspec,xltxtra,xunicode}
\defaultfontfeatures{Mapping=tex-text}
\setmainfont{Times New Roman}
\setsansfont[Scale=MatchLowercase,Mapping=tex-text]{Calibri}
\setmonofont[Scale=MatchLowercase]{Courier New}


\author{Xu Deyuan $<$\href{mailto:xudeyuanghw@gmail.com}%
            {xudeyuanghw@gmail.com}$>$}
%\date{}                                         % Activate to display a given date or no date

\newlength{\Han}
\settowidth{\Han}{汉}
\newcommand{\spreadCJK}[2]{\makebox[#1\Han][s]{#2}}

\begin{document}
\setlength{\headsep}{0cm}

\begin{flushright}
\parbox[c]{6em}{ %
{\small \makebox[\width][c]{学校代码:\quad10246 }\par
\makebox[\width][c]{学\phantom{占位}号:\quad13210200005}}}
\end{flushright}
\vspace{\stretch{0.5}}

\begin{figure}[htbp]
\begin{center}
\includegraphics[width=0.5\textwidth]{FudanLOGO.eps}
\end{center}
\end{figure}

\vspace{\stretch{0.5}}
\begin{center}
{\Huge \makebox [0.45\textwidth][s]{硕士学位论文}\par}

\vspace{\stretch{0.3}}
{\Large \makebox [0.2\textwidth][s]{(学术学位)}\par}

\vspace{\stretch{1.5}}
{\LARGE \textsf{抗弛豫镀膜气室中间接泵浦对Bell-Bloom共振的幅度和线宽的影响}\par}
\vspace{\stretch{0.5}}
{\Large \textrm{Effects of Indirect Pumping on the Amplitude and Linewidth of Bell-Bloom Resonance in Anti-relaxation Coated Cells}\par}

\vspace{\stretch{2}}
\parbox[c]{0.55 \textwidth}{%
\large \setlength{\baselineskip}{1.5\baselineskip}%
\spreadCJK{8}{学号}:\quad 13210200005\par
\spreadCJK{8}{姓名}:\quad 王梦冰\par
\spreadCJK{8}{专业}:\quad 原子与分子物理\par
\spreadCJK{8}{院系}:\quad 现代物理研究所\par}
%\spreadCJK{8}{姓名}:\quad 王梦冰\par
%\spreadCJK{8}{指导老师}:\quad 赵凯锋 \quad 副研究员\par

\vspace{\stretch{2}}
\parbox[c]{0.55 \textwidth}{\large \setlength{\baselineskip}{1.5\baselineskip}
\spreadCJK{8}{完成日期}:\quad 2016年10月1日}
\end{center}

\end{document}  