

\frontchapter{评审委员会}

\frontchapter{中文摘要}
\iffalse
原子磁力仪在测量时间为$T$,测量原子数为$N$,相干时间为$\tau$时的灵敏度为$\delta B=\frac{1}{g \mu_B}\frac{\hbar}{\sqrt{N \tau T}}$。其中$\mu_B$为玻尔磁子,$g$为基态的朗德$g$因子,$\hbar$为普朗克常数。增大基态的相干时间$\tau$的方法是减少弛豫,如壁弛豫,自旋碰撞弛豫等。在磁力仪信号上的表现则为更大的信号幅度和更小的展宽。
\fi
本文首先对双共振磁力仪和Bell-Bloom磁力仪的线型展开解析和数值计算,分别用到极化矢量模型和密度矩阵方法。计算结果表明,在两种磁力仪中,直接泵浦光都会产生功率展宽;在不考虑壁弛豫和自旋碰撞弛豫,而仅仅假设一个各项同性的弛豫机制下,间接泵浦光都不会产生功率展宽。本文的第二部分为实验研究,我们比较了OTS和石蜡气室中,在相同条件下,以及在同一气室中不同自旋交换速率下,间接泵浦对Bell-Bloom线宽和振幅的影响,并发现间接泵浦造成的功率展宽比直接泵浦小两个数量级,且信号幅度相当。另外间接泵浦功率展宽来自于壁碰撞或自旋交换碰撞对不同超精细能级上自旋去相干的传递。

\iffalse
我们在镀有OTS和石蜡抗弛豫镀膜的气室中,在不同温度下,即不同自旋交换碰撞率下,对Bell-Bloom磁力仪的线型进行详细研究。从实验结果上推测间接泵浦过程中泵浦光会通过自旋交换碰撞和壁碰撞传递功率展宽,且其引起的展宽相比直接泵浦小约两个数量级。
\fi

\bigskip
\noindent \textbf{关键词:\hspace{\Han}}
间接泵浦,Bell-Bloom磁力仪,功率展宽,自旋交换碰撞,壁弛豫

\bigskip
\noindent \textbf{中图分类号:\hspace{\Han}}
\frontchapter{Abstract}
In this thesis, we first derived the linewidth of the double resonance magnetometry using the polarization model and the density matrix method, analitically and numerically. We found no power-broadening effect from the indirect pumping light in both double resonance and Bell-Bloom magnetometry if ignoring effects of the wall and spin-exchange relaxation and assuming only isotropic relaxation process. In the second part of this thesis, we investigated the effect of indirect pumping on the lineshape of Bell-Bloom resonance in OTS and paraffin coated cells under the same condition, and at different spin exchange rates in the same cell. And found that the powerbroadening from the indirect pumping light is two orders of magnitude than that of the direct pumping with the comparable amplitude. The power broadening of indirect pumping light comes from the transfer of spin decoherence between different hyperfine levels.  
\iffalse
The sensitivity of a magnetic-field measurement for a magnetomery performed for a time $T$ with an ensemble of $N$ atoms with coherence time $\tau$ is $\delta B=\frac{1}{g \mu_B}\frac{\hbar}{\sqrt{N \tau T}}$, where $\mu_B$ is the Bohr magneton, g is the ground-state Landé factor, and $\hbar$ is Planck’s constant. To increase the coherence time $\tau$, one has to reduce the relaxation of polarization including the wall relaxation, spin-exchange relaxation and so on. So a larger amplitude of Lorentz signal with narrower linewidth is observed.
\fi

\bigskip
\noindent  \textbf{Key Words:\hspace{\Han}}
Indirect Pumping, Bell-Bloom Magnetometry, Power Broadening, Spin-exchange Collisions, Wall Relaxation

\bigskip
\noindent \textbf{CLC Number:\hspace{\Han}  }

